% Chapter 4

\chapter{分析和结论} % Main chapter title

\label{Chapter4} 
\section{数据分析}
\subsection{数据集和预处理}
测试数据集(见表\ref{table:data})是由中国联通智慧足迹数据科技有限公司所提供的从2017.9.1.-2017.11.30.的北京六环内的手机信令结果。 数据实验中选取从 2017.9.7.-2017.11.23.的数据为训练集,而之后的一周即2017.11.24-2017.11.30.的数据为测试集。
\begin{table}
\centering
\caption{数据集}
\label{table:data}
\begin{tabular}{p{0.3\columnwidth}|p{0.45\columnwidth}}
\hline
\hline
\textbf{Dataset} & \textbf{PopuBJ}\\
\hline
Data Type& Iphone Signal\\
\hline
Location & Beijing\\
\hline
Time Span & 9/1/2017-11/30/2017\\
\hline
Time interval & 1 hour\\
\hline
Grid map size & (53,54)\\
\hline

\end{tabular}
\end{table}
\subsection{模型比较}
将模型的预测结果和下面三种基线结果进行比较:
\begin{itemize}
	\item 插值方法:用历史平均来插值预测,对于某一ID,某一周,某一时刻,计算它的历史平均
	\item ARIMA: 前面进行初步训练的自回归积分滑动平均模型
	\item 多层感知机模型
\end{itemize}
\subsection{评估方法}
采取均方残差(RSME)的方法进行预测结果的精度分析,其数学表达式\cite{friedman2001elements}为:
\begin{equation}
R M S E = \sqrt { \frac { 1 } { z } \sum _ { i } \left( x _ { i } - \hat { x } _ { i } \right) ^ { 2 } }
\end{equation}
其中,$\hat{x_i}$和$x_i$分别是预测值和真实值,而$z$代表所有的数据的个数。
\section{预测结果}
\subsection{模型比较}
\subsection{参数的影响}
\subsection{结果可视化和区域分析}
